

\chapter{Indexation}

Par rapport au laboratoire précédent, la partie indexation n'est pas fondamentalement différente. Quelques points valent tout de même la peine d'être mentionnés.


\section{Changement des champs}

Selon la donnée du laboratoire, nous avons crééons quatres différents indexes. Ces derniers contiennent les champs \texttt{id} et \texttt{content} selon le listing \autoref{lst:index_fields}.

\begin{figure}[H]
\centering
\begin{lstlisting}	
// Add fields to document
if (item.id != null)
    doc.add(new Field("id", item.id, fieldType));

if (item.content != null)
    doc.add(new Field("content", item.content, fieldType));
\end{lstlisting}
\caption{Déclaration des champs dans les index}
\label{lst:index_fields}
\end{figure}

De plus, le champs \texttt{content} est composé des champs \texttt{title} et \texttt{summary} tel que dans le listing \autoref{lst:content}

\begin{figure}[H]
\centering
\begin{lstlisting}	
this.content = this.title+" "+this.summary;
\end{lstlisting}
\caption{Concaténation des champs title et summary}
\label{lst:content}
\end{figure}

\section{Analyzers différents}

Pour chacun des \texttt{Analyzers} :

\begin{itemize}
	\item \texttt{WhiteSpaceAnalyzer}
	\item \texttt{StandardAnalyzer}
	\item \texttt{EnglishAnalyzer}
	\item \texttt{EnglishAnalyzer} avec la liste personnalisée de stopwords
\end{itemize}

Nous avons créé et sauvegardé un indexe généré avec chacun de ces \texttt{Analyzers}.