\chapter{Interrogation}

La partie des requêtes à demandé un peu de travail. Nous avons du lire le fichier \texttt{query.txt} pour en extraire chacune des requêtes. Puis, dans une classe dédiée aux résultats, \texttt{Results}, nous avons créer une méthode pour exporter ceux-ci selon le même format que le fichier \texttt{qrels.txt}

De la même manière que pour l'indexation, l'interrogation des indexes a été fait avec les \texttt{Analyzers} correspondants. Au final, nous nous retrouvons avec quatre fichiers textes qui repsésentent les réponses aux 64 requêtes du fichier \texttt{query.txt}.