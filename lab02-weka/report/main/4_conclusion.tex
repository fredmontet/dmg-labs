\chapter{Conclusion}

En faisant ce laboratoire, nous avons découvert la puissance du logiciel Weka. Malgré une interface qui pourrait être plus intuitive, la possibilité de se connecter directement à une base de donnée MySQL permet de faire des analyses très rapidement. Ainsi, Weka semble être un logiciel très utile lorsqu'il s'agit de faire une analyse de donnée préliminaire ou, lorsqu'une contrainte de coût et/ou délai est présente. Un exemple d'utilisation relativement réaliste serait celui ou Weka est connecté à une base de donnée en production pour ajuster les différents public cible d'une gamme de produit.

En testant les différents types d'algorithmes que nous avons vu en cours, notre compréhension de ces derniers s'est améliorée par le biais d'un exemple pratique. Nous avons compris que certains algorithmes nécessites un paramétrage particulier pour donner de bon résultats de classification. Désormais, notre vision critique vis-à-vis des algorithmes à dispotion est plus claire.
