\chapter{Installation et configuration}

Dans notre situation nous avons décidé d'installer Weka sur une machine virtuelle Ubuntu 16.10.

\section{Installation}

Nous avons installé :

\begin{itemize}
	\item Java JDK 1.8\_111
	\item Weka 3.8.0
	\item Xampp x64 7.0.13
	\item MySQL connector for Java 5.1.39
\end{itemize}

Un élément à mentionner est que nous avons utilisé le connecteur dans sa version 5.1.39 alors que sur le site de MySQL\footnote{\url{https://dev.mysql.com/downloads/connector/j/}} il est disponible en 5.1.40. Ceci est dû au fait que nous avons utilisé la version disponible dans un repository ppa. L'installation à pu être faite avec la commande \texttt{sudo apt-get install libmysql-java}.

Hormis le point cité au paragraphe ci dessus, cette étape n'a pas posé de problème particulier qu'il serait nécessaire de mentionner. Il a suffit de suivre les différents tutoriaux relatifs à chacun des logiciels à installer.

\section{Configuration}

Pour configurer la machine virtuelle, un seul point a posé problème: la connexion à la base de donnée MySQL depuis Weka en utilisant le connecteur MySQL Java.

Le problème rencontré a été la prise en compte de la variable d'environnement \texttt{CLASSPATH} lors des tentatives de connexions. Sans cette variable, Weka ne trouve pas le connecteur.

Après avoir efféctué une recherche, nous nous sommes rendu compte qu'avec Ubuntu, il fallait préciser le \texttt{CLASSPATH} dans la commande pour démarrer Weka à l'aide du paramètre \texttt{-cp}. Le listing \autoref{lst:shell} présente le \texttt{.sh} qui a été utilisé pour démarrer Weka et le connecter à MySQL avec succès.

\begin{figure}[H]
\centering
\begin{lstlisting}
# Variables
java8=/home/fredmontet/Desktop/lab02-weka/asset/java/jdk1.8.0_111/jre/bin/java
weka=/home/fredmontet/Desktop/lab02-weka/asset/weka-3-8-0/weka.jar
connector=/usr/share/java/mysql-connector-java.jar

# Command
export CLASSPATH=$connector:$CLASSPATH
$java8 -Xmx300m -cp "$weka:$connector" weka.gui.GUIChooser
\end{lstlisting}
\caption{Contenu du fichier bash à exécuter pour lancer Weka}
\label{lst:shell}
\end{figure}
