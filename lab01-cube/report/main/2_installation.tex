\chapter{Installation}

L'infrastructure nécessaire pour l'utilisation d'un cube OLAP consiste en l'installation de 2 serveurs:

\begin{enumerate}
	\item MySQL
	\item icCube
\end{enumerate}

\section{MySQL}

L'installation du serveur MySQL a été simplifiée en utilisant la plateforme de développement web Wamp, réspectivement Mamp pour OSX ou encore Xamp pour Linux.
Ainsi, la configuration de ces outils est déjà faite et, de plus, des outils d'administration de base de donnée comme PhpMyAdmin y sont disponible.

\section{icCube}

L'installation du serveur OLAP icCube se fait de la même manière qu'une application commerciale grand publique, il suffit de suivre l'assistant d'installation. Sa seule dépendance est Java 1.8. 

Une fois l'installation faite, le serveur OLAP est disponible à l'adresse \texttt{\url{http://localhost:8282/icCube/icCube_en.html}}.
