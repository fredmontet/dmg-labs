\chapter*{Introduction}
\addcontentsline{toc}{chapter}{Introduction}

Ce laboratoire va nous permettre de mettre en pratique le contenu du cours de data warehousing et OLAP. Pour cela, nous allons utiliser un serveur de base de donnée MySQL et un serveur OLAP icCube.

La première partie du laboratoire consistera à l'installation et la configuration des serveurs cités ci-dessus. Dans un deuxième temps, nous allons explorer les possibilités fournies par un tel système à l'aide d'une série de requètes. Ces dernières seront composées des opérations typiques qu'il est possible de faire sur un cube OLAP (slice, dice, etc.).

