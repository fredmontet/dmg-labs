
\chapter{Conclusion}

Comme prévu, ce laboratoire de data warehousing et cube OLAP, nous a permis de mettre en pratique et de comprendree les notions vues en cours.

La création d'un cube s'est révélée aisée à l'aide du serveur OLAP icCube. Son interrogation, grace au language MDX\footnote{\url{https://en.wikipedia.org/wiki/MultiDimensional_eXpressions}}, permet d'obtenir des tableaux de résultats succins. Nous constatons donc l'utilité d'un tel outils pour faire de la Business Intelligence dans un contexte entrepreunerial. 

Concernant les différentes opérations OLAP, nous avons trouvé que les noms attribués à ces dernières: Drill down, Roll up, etc. conviennent bien à des cubes à 3 dimensions mais, lorsque le nombre de dimensions du cube $n$ dimensionnel augmentent, il devient plus difficile de se représenter les rotations du cube.