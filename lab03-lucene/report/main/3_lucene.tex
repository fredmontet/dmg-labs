\chapter{Comprendre l'API Lucene}

\section{Question 1}
\label{sec:question_1}
\textit{Does the command line demo use stopword removal? Explain how you find out the answer.}

Oui, les stop words sont supprimés avec la ligne de commande de la démo.

Dans Lucene, les \texttt{Analyzer} ont la tâche d'enlever les stop words. Étant donné qu'il y en a plusieurs types, il faut dans un premier temps savoir lequel est utilisé. 

En regardant la page de demo il est écrit : \texttt{The main()} method parses the command-line parameters, then in preparation for instantiating IndexWriter, opens a Directory, and instantiates StandardAnalyzer and IndexWriterConfig.
Le \texttt{StandardAnalyzer} est donc utilisé. Dans la documentation Lucene 6.3.0, on trouve que la classe textit{StandardAnalyzer}\footnote{\url{http://lucene.apache.org/core/6_3_0/core/org/apache/lucene/analysis/standard/StandardAnalyzer.html?is-external=true}} utilise le \texttt{StopFilter} avec une liste de stop words en anglais.

\section{Question 2}
\textit{Does the command line demo use stemming? Explain how you find out the answer.}

Non, le stemming n'est pas effectué avec la ligne de commande la demo.

Les \texttt{Analyzer} ont la tâche d'effectuer le stemming. Cependant, suivant lequel est utilisé, cela n'est pas le cas de tous. Comme cité dans la \autoref{sec:question_1} le \texttt{StandardAnalyzer} est utilisé. Selon la documentation le stemming n'est mentionné à aucun moment, on en déduit donc qu'il n'est pas effectué.

Notre deduction est comfortée en regardant le code source du \texttt{StandardAnalyzer}\footnote{\url{https://github.com/apache/lucene-solr/blob/master/lucene/core/src/java/org/apache/lucene/analysis/standard/StandardAnalyzer.java}} et en le comparant avec celui de l'\texttt{EnglishAnalyzer}\footnote{\url{https://github.com/apache/lucene-solr/blob/master/lucene/analysis/common/src/java/org/apache/lucene/analysis/en/EnglishAnalyzer.java}} à la ligne 107 qui lui, fait le stemming avec l'algorithme de Porter.

\section{Question 3}
\textit{Is the search of the command line demo case insensitive? How did you find out the answer?}

Non, la demo n'est pas sensible à la casse. 

On le constate très simplement en comparant le nombre de résultats de deux requêtes similaires. Nous avons efféctué plusieurs recherches, entre autre ``ANALYZER'' et ``analyzer''. Dans chacun des cas, le nombre de résultats obtenu à été le même pour les requêtes ont donnés le même nombre de résultat, respéctivement 522 pour la reququête citée ci-dessus. 

\section{Question 4}
\textit{Does it matter whether stemming occurs before or after stopword removal? Consider this as a general question.}

Oui, cet ordre joue un rôle dans le résultat.

Si le stemming est fait avant la suppression des stopwords, il est possible que certain mot de la liste des stopwords soient égaux à certains mots ``stemmés'' et donc, seront supprimés alors qu'à la base, ils ne sont pas des stopwords. Cet ordre n'est donc pas approprié. 

Dans le cas ou la suppression des stopwords est faite avant le stemming, on enlève d'abord les mots considérés comme inutiles et ensuite on ``stem'' ce sous-ensemble pour en trouver les racines en fonction de la langue désirée. C'est avec cet ordre que le résultat sera le plus juste.