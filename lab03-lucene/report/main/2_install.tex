\chapter{Installation de Lucene}

Pour installer Lucene, nous avons téléchargé son archive en version 6.3.0. Puis, nous avons suivi la page de demo \footnote{\url{http://lucene.apache.org/core/6_3_0/demo/overview-summary.html}}. Cette dernière nous a mené à un environnement fonctionnel de Lucene ou il est possible d'indexer une collection de document et d'effectuer des recherches dessus. 

Le listing \autoref{lst:lucene_classpath} montre les variables d'environnement ajoutées au \texttt{CLASSPATH}.

\begin{figure}[H]
\centering
\begin{lstlisting}	
# Lucene
export LUCENE_HOME="/Users/fredmontet/Dropbox/project/master/dmg/project/labs/lab03-lucene/app/lucene-6.3.0"
export LUCENE_CORE=${LUCENE_HOME}/core/lucene-core-6.3.0.jar
export LUCENE_ANALYZERS=${LUCENE_HOME}/analysis/common/lucene-analyzers-common-6.3.0.jar
export LUCENE_QUERYPARSER=${LUCENE_HOME}/queryparser/lucene-queryparser-6.3.0.jar
export LUCENE_DEMO=${LUCENE_HOME}/demo/lucene-demo-6.3.0.jar
export LUCENE=$LUCENE_CORE:$LUCENE_ANALYZERS:$LUCENE_QUERYPARSER:$LUCENE_DEMO

# Classpath
export CLASSPATH=$LUCENE:$CLASSPATH
\end{lstlisting}
\caption{Ajout au CLASSPATH}
\label{lst:lucene_classpath}
\end{figure}

Une fois la variable \texttt{CLASSPATH} mise à jour, il est possible d'utiliser Lucene et d'indexer 

\begin{figure}[H]
\centering
\begin{lstlisting}	

# Command to index
java org.apache.lucene.demo.IndexFiles -docs $LUCENE_HOME/docs

# Command to search
java org.apache.lucene.demo.SearchFiles 

\end{lstlisting}
\caption{Ajout au CLASSPATH}
\label{lst:lucene_classpath}
\end{figure}